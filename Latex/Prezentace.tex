\documentclass{beamer}
\usepackage{amsmath}
\usepackage{amsfonts}
\usepackage{amssymb}

\usepackage{graphicx}
\usepackage{subfig}

\usepackage[english]{babel}
\usepackage[utf8]{inputenc}

\usepackage{multicol}
\usepackage{units}

\usepackage{stmaryrd}
\usepackage{gensymb}
\usepackage{bibentry}
\usepackage{accents}

\usepackage{tensor}
\usepackage[bbgreekl]{mathbbol}
\usepackage{bm}
\usepackage{ulem}

\usepackage{hyperref}

\usetheme{CambridgeUS}
\setbeamercolor{item projected}{bg=red}
\setbeamertemplate{enumerate items}[default]
\setbeamertemplate{navigation symbols}{}
\setbeamercovered{transparent}

\setbeamercolor*{enumerate item}{fg=red}
\setbeamercolor*{enumerate subitem}{fg=red}
\setbeamercolor*{enumerate subsubitem}{fg=red}

\setbeamercolor{block title}{fg=red}
\setbeamercolor{caption name}{fg=red}

\title[Převrácené kyvadlo]{Zkoumání faktorů, které vedou ke kyvadlu kmitajícímu vzhůru nohama}
\author[J. P., L. K., M. F.]{J. Púček, L. Košárková, M. Fuksa}
\institute[Univerzita Karlova]{Univerzita Karlova, Česká republika}
\date{}

\input{vit-prusa-macros-experimental}

\let\newblock\relax

\begin{document}

\begin{frame}
\titlepage
\end{frame}

\section{Teoretický úvod}
\label{sec:uvod}

\begin{frame}
\begin{center}
			Naší zkoumanou diferenciální rovnicí bude rovnice~\eqref{eq:9} též zvaná Mathieuova rovnice
		\begin{equation}
			\label{eq:9}
			\ddd{\theta_{*}}{t_{*}}
			+
			\left(
			\alpha
			+
			\beta \cos \left( t_{*} \right)
			\right)
			\theta
			=
			0
			,
		\end{equation}
			pro nás konkrétně ve tvaru
		\begin{equation}
			\label{eq:10}
			\ddd{\theta}{t}
			+
			\left(
			\frac{g}{l}
			-
			\frac{A \Omega^2}{l} \cos \left( \Omega t \right)
			\right)
			\theta
			=
			0
			,
		\end{equation}
			kde $\alpha=\frac{g}{l\Omega^2}=\frac{\omega_{o}^2}{\Omega_{o}^2}$, $ \beta=-\frac{A}{l}$ a $t_{*}=\Omega t$. Za parametry jsme volili $g=9.81$, $l=1$,  $A=0.5$ a nefixní parametr $\Omega$.
		\end{center}
\end{frame}

\begin{frame}
	Z perturbační metody zjistíme podmínku stability kyvadla v horní poloze: 
	\begin{equation*}
		\frac{-\beta}{\sqrt{\alpha}}\geq \sqrt{2}
	\end{equation*}
	neboli:
	\begin{equation*}
	\frac{A}{l}\frac{\Omega}{\omega_{o}}\geq \sqrt{2}
	\end{equation*}	
\end{frame}

\section{Animace}
\label{sec:animace}

\begin{frame}
\begin{center}
	Volme například $\Omega=5$ (ostatní parametry volíme dle str.2), pak je nerovnost splněna:
\begin{equation*}
	3.19275 \geq \sqrt{2}
\end{equation*}	
	a podle teorie tyto podmínky vedou ke stabilizaci kyvadla v horní poloze, což potvrzuje i numerické řešení: 
\\

\href{run:./animace.mp4}{Převrácené kyvadlo - stabilní}
\end{center}
\end{frame}

\begin{frame}
\begin{center}
	Nyní naopak volme $\Omega=20$, pro které nerovnost splněna není:
\begin{equation*}
0.798189 \ngeq \sqrt{2}
\end{equation*}
	Podle teorie takto volená vstupní data nevedou k stabilizaci kyvadla v horní poloze. Numerickým řešením lze vidět, že teoretická předpověď nelže.
\\

\href{run:./animace2.mp4}{Převrácené kyvadlo - nestabilní}
\end{center}
\end{frame}




\end{document}




