\documentclass{beamer}
\usepackage{amsmath}
\usepackage{amsfonts}
\usepackage{amssymb}

\usepackage{graphicx}
\usepackage{subfig}

\usepackage[english]{babel}
\usepackage[utf8]{inputenc}

\usepackage{multicol}
\usepackage{units}

\usepackage{stmaryrd}
\usepackage{gensymb}
\usepackage{bibentry}
\usepackage{accents}

\usepackage{tensor}
\usepackage[bbgreekl]{mathbbol}
\usepackage{bm}
\usepackage{ulem}

\usepackage{hyperref}

\usetheme{CambridgeUS}
\setbeamercolor{item projected}{bg=red}
\setbeamertemplate{enumerate items}[default]
\setbeamertemplate{navigation symbols}{}
\setbeamercovered{transparent}

\setbeamercolor*{enumerate item}{fg=red}
\setbeamercolor*{enumerate subitem}{fg=red}
\setbeamercolor*{enumerate subsubitem}{fg=red}

\setbeamercolor{block title}{fg=red}
\setbeamercolor{caption name}{fg=red}

\title[Převrácené kyvadlo]{Zkoumání faktorů, které vedou ke kyvadlu kmitajícímu vzhůru nohama}
\author[J. P., L. K., M. F.]{J. Púček, L. Košárková, M. Fuksa}
\institute[Univerzita Karlova]{Univerzita Karlova, Česká republika}
\date{\today}

\input{vit-prusa-macros-experimental}

\let\newblock\relax

\begin{document}

\begin{frame}
\titlepage
\end{frame}

\section{Teoretický úvod}
\label{sec:uvod}

\begin{frame}
\begin{equation*}
  \ddd{\theta}{t}
  +
  \left(
    \frac{g}{l}
    -
    \frac{A \Omega^2}{l} \cos \left( \Omega t \right)
  \right)
  \theta
  =
  0
  .
\end{equation*}
\end{frame}

\section{Animace}
\label{sec:animace}

\begin{frame}
\begin{center}
\href{run:./animace.mp4}{Převrácené kyvadlo - stabilní}
\begin{equation*}
3.19275 \geq \sqrt{2}
\end{equation*}
\end{center}
\end{frame}

\begin{frame}
\begin{center}
\href{run:./animace2.mp4}{Převrácené kyvadlo - nestabilní}
\begin{equation*}
0.798189 \ngeq \sqrt{2}
\end{equation*}
\end{center}
\end{frame}




\end{document}




